\section{Determination of the $\chi_b$ yields}
\label{sec:ChibFit}

The yields of $\chi_b(1P, 2P, 3P)$ \to \OneS $\gamma$ are determined with fits
to the $m(\mu^+\mu^-\gamma) - m(\mu^+\mu^-)$ invariant mass difference. 
All fits are applied in the mass difference range from 0.32 to 1.4 \gevcc.
Combinatorial background is described by exponential function multiplied by the 
polynomial:

\begin{equation}
    \label{eq:chibfit_bg}
    e^{-\tau x}(a_1x+a_2x^2+...+a_nx^n)
\end{equation}
As shown in Table~\ref{tab:chibfit_order}, the order of the background polynomial ($n$) depends on $p_{T}^{\OneS}$ interval.

\begin{table}[t]
  \caption{
    \small The order of background polynomial
    }
    \centering
   \begin{tabular}{rc}
    \hline
    $p_{T}^{\OneS}$ interval & Polynomial order ($n$)\\ 
    \hline
    10 --- 12 \gevc & 5 \\
    12 --- 18 \gevc & 3 \\
    18 --- 30 \gevc & 2 \\
    \hline
  \end{tabular}
\label{tab:chibfit_order}
\end{table}


The separation of $\chi_{b0,2,3}$ signals is a challenge  since 
$\chi_{b0, 1, 2}$ masses are located very close and the detector resolution 
is not enough to disjoint them. Thus only three signal peaks
can be visually observed in the mass difference plot. These peaks corresponds
to \chibOneP, \chibTwoP, \chibThreeP signals and are composed of the 
$\chi_{b0,1,2}$ signals.

The knowledge of the structure of each $\chi_{b}(1P,2P,3P)$ signal peak is used in the
fit model to construct the signal function. $\chi_{b}(jP)$ signal peak is described
by the sum of two CrystalBall functions that corresponds to \chibone and \chibtwo
signals. In this study \chibzero signal did not take into account since
 \chibzero decay modes have a low branching ratio. Thus the signal function is
a sum of six CrystalBalls:

\begin{equation}
    \label{eq:chibfit_bg}
    \sum_{i=1}^{2}\sum_{j=1}^{3}{CB_{\chi_{bi}(jP)}(\Delta M; \alpha_{\chi_{bi}(jP)}, n_{\chi_{bi}(jP)}, \mu_{\chi_{bi}(jP)},
    \sigma_{\chi_{bi}(jP)})}
\end{equation}

In the signal function the free parameters are $\mu_{\chi_{b1}(1,2,3P)}$ and $\sigma_{\chi_{b1}(1P)}$.
The values of $\mu_{\chi_{b2}(1,2P)}$ are constrained by the $\mu_{\chi_{b1}(1,2P)}$ value and the corresponding mass difference taken from PDG: 
 $\mu_{\chi_{b2}(jP)} = \mu_{\chi_{b1}(jP)} + \Delta M_{\chi_{b1,2}(jP)}^{PDG}$, j=1,2. The $\sigma_{b1}(2,3P)$ parameters are constrained by the 
  $\sigma_{\chi_{b1}(1P)}$ value and the ratio obtained from Monte Carlo fits: $\sigma_{\chi_{b1}(jP)} = \sigma_{\chi_{b1}(1P)}\frac{\sigma_{\chi_{b1}(jP)}^{MC}}{\sigma_{\chi_{b1}(1P)}^{MC}}$  
The \sigma_{\chi_{b2}(jP)} values is taken equal to \sigma_{\chi_{b2}(2P)}. 

 The following parameters are fixed equal: $\sigma_{\chi_{b1}(jP)}=\sigma_{\chi_{b2}(jP)}$ j=1,2,3;


$\alpha_{\chi_{b1}(1P)}=\alpha_{\chi_{b2}(1P)}=-1.1$;  $\alpha_{\chi_{b1}(2P)}=\alpha_{\chi_{b2}(2P)}=-1.23$;  $\alpha_{\chi_{b1}(3P)}=\alpha_{\chi_{b2}(3P)}=-1.8$; $\n_{\chi_{b1}(1P)}=\n_{\chi_{b2}(3P)}=-1.8$
% The position of 

% As was mentioned above the detector resolution is not enough to disjoint \chibone
% and \chib signals. Therefore the constraints are applied on CrystalBall
% parameters and on the  \chibone and \chibtwo signal yields:
% \begin{enumerate}
% \item The position of \chibtwo peak ($\mu_{\chi_{b2}(jP)}$) is fixed in accordance with 
% \end{enumerate}